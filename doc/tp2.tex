\documentclass[
	12pt,
	a4paper,
	onepage,
	brazil
]{article}

\usepackage[brazil]{babel}
\usepackage[utf8]{inputenc}
\usepackage[T1]{fontenc}
\usepackage{lmodern}
\usepackage{hyperref}
\usepackage{amsmath}
\usepackage{amsthm}
\usepackage{amsfonts}
\usepackage{indentfirst}
\usepackage[most]{tcolorbox}
\usepackage{inconsolata}
\usepackage{caption}
\usepackage{floatrow}
\usepackage[lined,algonl,ruled]{algorithm2e}
\usepackage{float}
\usepackage{graphicx,url}
\usepackage{times,epsfig}
\usepackage[
	backend=bibtex8,
	style=numeric
]{biblatex}
\addbibresource{tp2.bib}
\usepackage{xcolor}
\hypersetup{
	colorlinks,
	linkcolor={red!50!black},
	urlcolor={red!80!black}
}

\usepackage[
	lmargin = 25mm,
	rmargin = 25mm,
	tmargin = 25mm,
	bmargin = 25mm
]{geometry}

\sloppy

\author{Pedro Otávio Machado Ribeiro}
\title{Trabalho Prático 2\\Índice Invertido\\Algoritmos e Estruturas de Dados III - 2017/01}
\date{14/06/2017}

\begin{document}

	\maketitle
	
	\section{Introdução}
	
	Neste trabalho temos como contexto o rapaz Hetelberto Topperson que gosta muito de conversar no \textit{ZipZop}, porém ele possui uma memória seletiva e consegue lembrar do assunto que conversou com uma pessoa, mas não da pessoa. Hetelberto gostaria que o \textit{ZipZop} permitisse buscas que retornem o trecho de uma conversa com alguém dado um uma palavra qualquer. Dado isso, Hetelberto pediu ajuda para construir um índice invertido, já que sua amiga Inês implementara o buscador.
	
	O problema consiste em, dado $D$ arquivos de conversa de Hetelberto presentes num diretório $E$ e $M$ bytes de primária disponível, devemos escrever o índice invertido cada palavra de todos os arquivos em $M$ no arquivo \textit{index} dentro do diretório $S$. A solução do problema se resume a ordenar as palavras e obter os parâmetros que compões o índice invertido. Para resolver este problema, os seguintes são necessários:
	
	\subsection{Ordenação Interna}
	
	Dado que para realizar uma ordenação externa com um limite de memória primária dado, é necessário saber como ordenar valores armazenados na memória primária. Para isso, escolhi o \textit{Quicksort} para ordenar os dados. Para uma breve descrição sobre este método, veja \cite{quicksort-wiki}.
	
	\subsection{Índice Invertido}
	
	O Índice Invertido, conhecido também como índice remissivo, é conhecido popularmente como aquela lista no final de livros, artigos, etc, de cunho acadêmico/informativo de palavras seguidas de números refernetes às páginas em que estas podem ser encontradas.
	
	\section{Metodologia}
	
	\section{Complexidade}
	
	\section{Experimentos}
	
	\section{Análise de Resultado}
	
	\section{Conclusão}
	
	\nocite{*}
	
	\printbibliography[title=Referências]

\end{document}